%
%  revised  front.tex  2011-09-02  Mark Senn  http://engineering.purdue.edu/~mark
%  created  front.tex  2003-06-02  Mark Senn  http://engineering.purdue.edu/~mark
%
%  This is ``front matter'' for the thesis.
%
%  Regarding ``References'' below:
%      KEY    MEANING
%      PU     ``A Manual for the Preparation of Graduate Theses'',
%             The Graduate School, Purdue University, 1996.
%      TCMOS  The Chicago Manual of Style, Edition 14.
%      WNNCD  Webster's Ninth New Collegiate Dictionary.
%
%  Lines marked with "%%" may need to be changed.
%

  % Dedication page is optional.
  % A name and often a message in tribute to a person or cause.
  % References: PU 15, WNNCD 332.
\begin{dedication}
  To my wife and my parents. I couldn't have done this without you.
\end{dedication}

  % Acknowledgements page is optional but most theses include
  % a brief statement of apreciation or recognition of special
  % assistance.
  % Reference: PU 16.
\begin{acknowledgments}
  It has been a long journey, and I could not have been where I am now without the help
  and support of many people.

  I would like to thank the other committee menbers, Dr. Ryan Hafen, Dr. 

  I want to express my appreciation to my former and present colleagues, Jianfu, Xiang, 
  Ashrith, Philip, Aritra, Yuying, Qi, Barret, for all the head-scratching and laughters
  we have shared while computing and plotting with data.

  I also want to thankn all my dear friends at Purdue, who have made these years most
  enjoyable and memorable. Cheng, thank you for being the best mentor from the first 
  day I came to the department; Cheng and Ximing, thank you for all the fun 
  we have had as roommates and ever since; Qiming, Yang, Longjie, Xiaoguang, Hanli, 
  Zhuo, Shuai, Faye, April, Kelly-Ann, thank you for being such an awesome class of 2010;
  this list goes on and on, and I thank you all for spending your time with me.

  I am very grateful to the Department of Statistics for providing an excellent 
  environment, various opportiunities, and a diverse group of people. The quality of
  my education here has been exceptional, and I have learned so much from many 
  faculty members and fellow students. Sincere thanks to Dr. Rebecca Doerge for all
  the help and support during this six years. And a special thank goes to Doug for all
  the ehlp and patience on my projects.

  Last, but certainly not least, I want to thank my parents, and my beloved wife, for
  their unconditional love, patience, and encouragement, and for making me a home 
  that I can always return to. I owe you so much.
\end{acknowledgments}

  % The preface is optional.
  % References: PU 16, TCMOS 1.49, WNNCD 927.
%\begin{preface}
%  This is the preface.
%\end{preface}

  % The Table of Contents is required.
  % The Table of Contents will be automatically created for you
  % using information you supply in
  %     \chapter
  %     \section
  %     \subsection
  %     \subsubsection
  % commands.
  % Reference: PU 16.
\tableofcontents

  % If your thesis has tables, a list of tables is required.
  % The List of Tables will be automatically created for you using
  % information you supply in
  %     \begin{table} ... \end{table}
  % environments.
  % Reference: PU 16.
\listoftables

  % If your thesis has figures, a list of figures is required.
  % The List of Figures will be automatically created for you using
  % information you supply in
  %     \begin{figure} ... \end{figure}
  % environments.
  % Reference: PU 16.
\listoffigures

  % List of Symbols is optional.
  % Reference: PU 17.
\begin{symbols}
  $m$& mass\cr
  $v$& velocity\cr
\end{symbols}

  % List of Abbreviations is optional.
  % Reference: PU 17.
\begin{abbreviations}
  abbr& abbreviation\cr
  bcf& billion cubic feet\cr
  BMOC& big man on campus\cr
\end{abbreviations}

  % Nomenclature is optional.
  % Reference: PU 17.
\begin{nomenclature}
  Alanine& 2-Aminopropanoic acid\cr
  Valine& 2-Amino-3-methylbutanoic acid\cr
\end{nomenclature}

  % Glossary is optional
  % Reference: PU 17.
\begin{glossary}
  chick& female, usually young\cr
  dude& male, usually young\cr
\end{glossary}

  % Abstract is required.
  % Note that the information for the first paragraph of the output
  % doesn't need to be input here...it is put in automatically from
  % information you supplied earlier using \title, \author, \degree,
  % and \majorprof.
  % Reference: PU 17.
\begin{abstract}
  This is the abstract.
\end{abstract}
