\chapter{BACKGROUND}

This chapter introduces local regression and some time series modeling methods,
with emphasis on a time series decomposition method: seasonal trend decomposition
using loess (STL), and a spatially local regression method. The goal is to be
brief but still cover the methods thoroughly, as much of the new work developed
later depends heavily on these topics.

\section{Nonparametric Regression}


\subsection{Local Regression Models}

Often nature presents dependencies that are much more complex than can be captured
with a simple parametric model. When the data suggest that $\mu(x)$ is a complex
function, one may try to accommodate this by specifying a complex parametric model
with many parameters.

Another approach to modeling a complex $\mu$ that is much more simple and flexible
is local regression \cite{Cleveland:1979}. Instead of constraining $\mu(x)$ to 
have a parametric form, assume that the data locally, around some neighborhood of
x, can be well-approximated by a parametric form, namely low order polynomials.

The same idea can be illustrated when replacement of a quantitative variable by
indicator variables 

\subsection{Fitting a Local Regression Model}


\subsection{Computation}

\section{Seasonal Trend Decomposition using Loess (STL)}

Seasonal trend decomposition using loess (STL) is a method for decomposing a time
series into seasonal, trend, and remainder components \cite{Cleveland:1990}.

\subsection{STL with Added Feature}

\section{Divide and Recombine (D\&R) for Large Complex Data}

\subsection{D\&R Statistical Framework}

D\&R \cite{Guha:2012} is a statistical framework for the analysis of large complex
data that enables feasible and practical analysis of large complex data. 

The 
analyst selects a division method to divide the data into subsets, applies an 
analytic method of the analysis to each subset independently with no communication
among subsets, selects a recombination method that is applied to the outputs 
across subsets to form a result of the analytic method for the entire data.

Analytic methods have two distinct categories, visualization methods whose outputs
are visual displays, and number-category methods whose outputs are numeric and 
categorical values. In D\&R, number-category analytic methods are typically applied
to each of the subsets. Visualization methods are typically applied to each subset 
in a sample of subsets because often there are too many of them to look at plots 
of all \cite{Hafen:2013}.
 
\subsection{Computational Environment}

The front end of our computational environment is R \cite{R}, a widely used 
software environment for statistical computing and graphics. On the other side, the
back end is the Hadoop which consists of Hadoop Distributed File 
System (HDFS) \cite{HDFS} for storage and processing engine (MapReduce) 
\cite{mapreduce}. RHIPE \cite{Guha:2010}, the R and Hadoop Integrated Programming 
Environment, bridges the gap between these two ends. 

As analyst, we only need to specify R commands to carry out a D\&R job which
consists of following three steps:
\begin{itemize}
\item Divide the whole dataset into subsets. It can be randomly dividing or
can be done conditional on a given categorical variable. For example, for a spatial
temporal data set, we can either divide the whole data by time or by location. 
\item Apply the analytic method to each subset parallelly.   
\item recombine the outputs of the A computations and write results to the HDFS. 
\end{itemize}  

The first step can be achieved by one MapReduce job using RHIPE R commands which 
creates the subsets from the original raw data set sat on HDFS and 
distributes the subsets across the servers of the cluster onto the Hadoop 
Distributed File System (HDFS) as key-value pairs. Most of situation, the original
raw data can be a raw text file saved on HDFS.

Thereafter, the second and
third steps can be implemented with another MapReduce job also specified by
analyst in RHIPE R commands. In this second MapReduce job, a group of Map
computation procedures are running embarrassingly parallel with a free core 
assigned to each, which means independently with no communication among those Map
procedures. We call those Map procedures the Mapper. Then, a data block or a 
collection of subsets will be passed into those Mappers, and each subset
will be applied with the analytic method independently.

