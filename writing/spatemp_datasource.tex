\section{Source of the Data}

The data is a compendium of different levels of weather data ranging from stations
taking regular hourly measurements, such as those at airports, to cooperative 
observer stations where the records may only include daily values, have gaps in 
time or might not measure both temperature and precipitation. The original source
for the data are the data archives at the National Climatic Data Center 
(http://www.ncdc.noaa.gov) although these data have been further processed to 
combine stations at similar locations and eliminate stations with short records

Historical records of weather such as monthly precipitation and temperatures from 
the last century are an invaluable database to study changes and variability in 
climate. These data also provide the starting point for understanding and modeling 
the relationship among climate, ecological processes and human activities. 

Because of digitization and rapid development of computer hardware, nowadays
people are able to store, summarize and analyze larger and more 
complex climate data set than before, which stimulates the dramatically 
increasing  of the demand and interest of high-value environmental data and 
information. In United States, National Centers for Environmental Information, 
formerly the National Climatic Data Center (NCDC), is responsible for hosing and 
providing access to one of the most significant archives of environmental 
information. Some of data set hosted by NCEI \cite{NCEI} are shown as follows:

\begin{itemize}
  \item COOP:
    Through the National Weather Service (NWS) Cooperative Observer Program 
    (COOP), more than 10,000 volunteers take daily weather observations at 
    National Parks, seashores, mountaintops, and farms as well as in urban and 
    suburban areas. COOP data usually consist of daily maximum and minimum 
    temperatures \cite{COOP}.
  \item SNOTEL:
    The Natural Resources Conservation Service (NRCS) operates and maintains an 
    extensive and automated system (SNOwpack TELemetry or SNOTEL) designed to
    collect snowpack and related climatic data like air temperature in the Western
    United States begins in 1978 \cite{SNOTEL}.
  \item AG:
    Agricultural climate data for southeastern Washington from United States
    Department of Agriculture Natural Resources Conservation Service (USDA-NRCS) 
    \cite{USDA}.
  \item MRCC:
    Midwest Climate Data Center data, mainly for period between 1895 and 1948. 
    The MRCC \cite{MRCC} serves the nine-state Midwest region (Illinois, Indiana, 
    Iowa, Kentucky, Michigan, Minnesota, Missouri, Ohio, and Wisconsin).
  \item USHCN:
    Historical Climate Network data provides summary of the month temperature and 
    precipitation observations for 1,218 stations across the contiguous United 
    States. Temperature observations have been homogeneity corrected to remove 
    biases associated with non-climatic influences, such as changes in 
    instrumentation and observing practices, and changes to the environment 
    including station relocations \cite{USHCN}.
\end{itemize}

In 2002, based on the data set we listed above, the NCEI has further
processed them to produce a consolidated and uniform 103-year spatial temporal
data set by combining stations at similar locations, eliminating stations with
short records, and aggregating some of daily measurement to be monthly.
In summary, the data we are going to analyze is about observed monthly average
maximum daily temperatures for the conterminous US from 1895 to 1997. There are
8,125 stations reporting monthly average maximum daily temperatures at some time
in this period which includes 1,236 months. 