\chapter{Multi-factor designed experiment for performance of the 
Nonparametric-Regression Modeling}

In this chapter, we shift our gears to the analysis of the performance of the 
analysis method we proposed in previous chapter for a large spatial temporal data.
There are two groups of tunning parameters for our nonparametric analysis model.
One is the tunning parameters of statistical model, such as smoothing window, 
smoothing degree for both temporal fitting and spatial fitting, which we have
already demonstrated in \cite{} and \cite{} using cross-validation method. Another
group of tunning parameters are user-tunable MapReduce tunning parameters. Within
this chapter, we illustrate a full factorial experiment to study the affect of
these system tunning parameters to the modeling.

\section{Type of MapReudce Job}

There are three main steps in one MapReduce job: Map, Shuffle and Sort, Reduce.

Within our routine of analysis for spatial-temporal data, we categorize all 
necessary MapReduce job into two different groups. The first type of MapReduce
job is mainly focus on one of smoothing models, which is in either spatial or 
temporal dimension. We name it as $model$-$fitting$ job. For this type of job, it 
reads in a given type of division (either by time or by location) from HDFS, and 
then carries out one type of smoothing procedure, depends on what the division is,
within its Map step. There is no need for neither Reduce step nor shuffle and sort
step in this type of job since the division format is not changed. So the output 
of Map function is directly written onto HDFS.

Another type of job is focusing on switching from one division to another, which
we name as $swapping$ job. For this type of job, it reads in a given type of 
division from HDFS. And for each input key-value pair, the Map function generates
multiple intermediate key-value pairs with different keys. For instance, if by 
time division is read in, and observations of all locations in each month is one 
input key-value pair, then 7,738 intermediate key-value pairs is created with each
location ID as key in the Map function of $swap$-$to$-$location$ MapReduce job.
Then the output of Map function is first written to local disk of nodes 

\section{Modeling Routine}
\label{sec:routine}
The whole analysis routine consist of six MapReduce jobs, which are listing as
following in order:

\subsection{Reading in}
\label{sec:readin}
The first MapReduce job in the modeling routine is the Reading in job.
The raw text data files as mentioned in section~\ref{sec:Download}, which is 
sitting on HDFS, is first merged into one text file, and then duplicated by
row. Specifically, the time series of monthly maximum temperature of each station
is replicated based on original 48 years observations to include $2^{16}$ years 
which is $786,432$ monthly observations.
The total size of the text data file is 33 GB which includes monthly maximum
temperature observation over $786,432$ months at 7,738 different locations.

\begin{description}
  \item[Input] The duplicated raw text data file on HDFS is read in as input. Each
  row is read in as a key-value pair. The key is a unique row index, and value
  is the corresponding string in that row. 
  \item[Output] The value of each output key-value pair is a matrix with dimension
  7,738 by 2. Each row represents the maximum temperature and station id of one 
  location for a given month, and the key is the corresponding month index varies 
  from 1 to $786,432$. 
  \item[Map] Every block of the raw text data saved on HDFS is sent to one of 
  mapper with each row as one of input key-value pairs. By calling the R function
  \texttt{strsplit}, each filed of the row string is split apart. Since each
  row contains 12 monthly maximum temperature of a given year at a given location,
  we generate a intermediate key-value pair for each of them for all locations
  and all years. The key of 
  intermediate key-value pairs is the month index of corresponding row, and the 
  value is a vector with station id and maximum temperature.
  \item[Reduce] All intermediate key-value pairs that share the same month index
  are shuffled and sent to one Reducer. By calling R function \texttt{rbind}, 
  all intermediate values are row-binded into one matrix with dimension of 7,738
  by 2. Finally there are $786,432$ key-value pairs generated and saved on HDFS.
\end{description}

Notice that the computation complexity of each Mapper is very light, mainly just
\texttt{strsplit} and \texttt{rhcollect} function are called. On the other hand,
the size of output data of each Mapper is roughly close the input data. In other
words, a lot of intermediate key-value pairs, in term of size, are shuffled and
sorted and then sent to corresponding Reducers. For this type of job, the best 
performance can be obtained by assigning more memory from heap size of each JVM 
to use for shuffle and sorting, which can avoid multiple meaningless I/O to disk.  
On the reduce side, it is also very light with respect to computation. Only 
\texttt{rbind} function is called repeatedly. So the best performance is obtained 
when the intermediate data can reside entirely in memory of the Reducer JVM, 
which achieved by assigning more memory from the JVM of Reducer for the purpose
of holding as many as possible intermediate key-value pairs in memory. 

\subsection{Spatial Smoothing of Original Observation}
\label{sec:spaofit}
By taking the subset by time, the second 
MapReduce job is applying spatial loess smoothing fitting to each month 
independently in parallel. Meanwhile, a RData file which contains meta-data such
as longitude, latitude, and elevation about all 7,738 stations are saved on HDFS.
These information is used for each spatial loess smoothing in the Map step. 

\begin{description}
  \item[Input] $786,432$ key-value pairs generated in the first reading in job.
  The key is the month index, and the value is a matrix with dimension of 7,738
  by 2. First column is location index, and another column is the corresponding 
  maximum temperature observation.
  \item[Output] The number of output key-value pairs is still $786,432$. The key
  is still the month index, but the value is a vector of spatial loess fitted
  value with length 7,738, one for each location. The order of fitted value in 
  each vector is kept as same based on the order of locations. So we can get rid
  of the location information from the output key-value pairs.
  \item[Map] The shared RData file containing all location information is first 
  copied and load into the global environment of each Mapper. The input matrix
  is merged with station information R object (a data.frame with 7,738 rows and
  \texttt{lon}, \texttt{lat}, and \texttt{elev} columns). Then \texttt{spaloess}
  function from package \texttt{Spaloess} is called to calculate the spatial
  loess smoothing value for each station at given month. After the fitting,
  all spatial information of each location is dropped. The key of each 
  intermediate key-value pair is month index, and value is vector of spatial 
  smoothing value in the same order of locations. 
  \item[Reduce] There is no Reduce need for this job. The intermediate key-value
  pairs are directly written to HDFS.
\end{description}

This particular MapReduce job is different than the first reading in job. Here
the Reduce step is not necessary. Moreover, the shuffle and sorting stage should
also be avoid to save unnecessary network traffic and multiple trips to the local 
disk, and the final results of Map are directly wrote to HDFS. Consequently, the
MapReduce tuning parameters which affect the shuffle and sorting stage are not 
needed to be considered in this job. The best performance of this job can be 
obtained by enlarging the memory assigned to each Mapper for the computation.

\subsection{Swapping from By-time Division to By-location Division}
\label{sec:swaptoloc}
After the spatial smoothing in each month, we have to switch the data from by 
month division to by location division in order to proceed the smoothing procedure
in time dimension. 

\begin{description}
\item[Input] $786,432$ key-value pairs. Key is the month index, and value is a 
vector with length of $7,738$. The order of the numeric values in each vector are 
the same in all $786,432$ key-value pairs.
\item[Output] $7,738$ output key-value pairs are generated. The key is changed
to be location index, and the corresponding value is matrix with dimension 
$786,432$ by 2. One column is the month index, and another column is the spatial
smoothed value.
\item[Map]An intermediate key-value pair is generated for each element of the
value of each input key-value pair. Totally, there are $786432 \times 7738$ 
intermediate key-value pairs generated after the Map. The intermediate key is the 
index of the element which is the location index, and the corresponding value is 
a vector with two numbers, month index which is the input key, and the element 
itself.   
\item[Reduce] All intermediate key-value pairs that belong to the same location
are sorted, shuffled, and sent to one Reducer. In the Reduce, these values of
vector with length two are \texttt{rbind} together to be a matrix with dimension
$786,432$ by 2.
\end{description}

Similar with the first \textbf{Reading in} MapReduce job, this job has very light
computation in both Map and Reduce. But it does require heavy system I/O because
the size of intermediate output from Map is about the same size as input files.
So the network traffic between Map and Reduce and disk I/O in the Map are also 
very heavy. The best performance with respect to the running time for this job
should be achieve by assigning more memory of heap size of the Mapper's or 
Reducer's JVM to the shuffle and sort stage of the job, in order to avoid 
unnecessary strips to the local disk.

\subsection{Temporal Fitting}
\label{sec:stlfit}
The fourth MapReduce job in the routine is mainly focus on the temporal fitting
on the time series at each location. This job is similar with the second job in
the routine which proceeds the spatial loess smoothing on the original observation
at each time point. Only Map stage is necessary, shuffle and sort stage and even
reduce stage can be avoid. The outputs of Map are directly wrote to HDFS.

\begin{description}
\item[Input] $7,738$ key-value pairs. Key is the location index, and value is a 
matrix with dimension $768,432$ by $2$. Two columns are month index and spatial
smoothed value.
\item[Output] Still $7,738$ output key-value pairs are generated. The key is kept
as location index, however the corresponding value is updated to a matrix with 
four columns of trend, seasonal, month index, and spatial smoothed value. 
\item[Map] For each input key-value pair, \texttt{stlplus} function is called
with pre-specified temporal smoothing parameters on the spatial smoothed time
series. Two new columns therefore are generated and added to the matrix of input
value. 
\item[Reduce] There is no Reduce needed for this job, the outputs of map are 
directly written to HDFS.
\end{description}

This is a job with only map stage. Shuffle and sorting and reduce are all avoid 
in this job. Therefore the MapReduce tunning parameters we are considering do
not have any effect to the performance of this job in term of running time.

\subsection{Swapping from by location division to by time division}
\label{sec:swaptotime}
Once the temporal fitting is done in the previous MapReduce job, the by location
division with all temporal fitting results, such as trend, seasonal components,
and remainder is read into a new MapReduce job to generate the by time division,
which will be used for further spatial fitting.

\begin{description}
\item[Input] $7,738$ key-value pairs. Key is the location index, and value is a 
matrix with dimension $768,432$ by $4$, which includes temporal fitting results.
\item[Output] $768,432$ output key-value pairs are generated. The key is changed
to month index, the corresponding value is updated to a matrix of $7,738$ rows 
and four columns which are trend, seasonal, location index, and spatial smoothed 
value. 
\item[Map] For each input key-value pair, a intermediate key-value pair is 
collected for each row of the matrix value. Therefore, $768,432$ intermediate
key-value pairs are created from each input key-value pair. Totally it is 
$5,946,126,816$ key-value pairs, whose value is just a one-row matrix.
\item[Reduce] Intermediate key-value pairs that share the same month index are 
sorted and sent to one Reduce, then merged together by calling \texttt{rbind}
function. The final by time division is saved on HDFS.
\end{description}

In this MapReduce job, computational complexity and memory usage requirement are 
quite light in both Map and Reduce stage. But more memory should be allocated for
shuffle and sorting stage of this job since the size of data is not shrunk by
the Map function. Hadoop parameters for shuffle and sort stage should be tunned 
to avoid multiple trips to local disk during this stage. 

\subsection{Spatial Fitting of Remainder}
\label{sec:sparfit}
The last step for fitting is the spatial fitting for the remainder component from
the STL smoothing. This is done in a MapReduce job, and input and output of this
job are both by time division. The job does not change the key of input key-value
pairs.

\begin{description}
\item[Input] $768,432$ key-value pairs are read in. Key is the month index, and 
value is matrix with dimension $7,738$ by $4$, which includes temporal fitting 
results.
\item[Output] $768,432$ output key-value pairs are generated. The key is kept as
month index, the corresponding value is updated to a matrix of $7,738$ rows 
and five columns which includes a new column of spatial fitted value of remainder.
\item[Map] The shared RData file containing all location information is copied
load into the global environment of each Map. The the input value is merged with
location information by location index. The \texttt{spaloess} function is called
on the column of remainder for each input value. After the fitting, the location
information with longitude, latitude and elevation are removed from the matrix
when collect as intermediate value in order to illuminate the size of the data
written to local disk.
\item[Reduce] Reduce is not needed in this job. The outputs of Map are directly
written to HDFS as by month division.
\end{description}

\subsection{By location division of Spatial Fitting of Remainder}
\label{sec:swaptoloc2}
The final step of the entire modeling routine is a MapReduce job to generate by
location division with all estimated components from by time division. It is a
job with Map, shuffle and sorting, Reduce stage.

\begin{description}
\item[Input] $768,432$ key-value pairs of by month division. Key is the month 
index varying from 1 to $768,432$, and value is a matrix with dimension $7,738$ 
by $4$, which includes all temporal fitting components, the spatial smoothed value
of original observation, and the spatial smoothed value of remainder.
\item[Output] $7,738$ output key-value pairs are generated. The key is changed
to location index, the corresponding value is restructured to be a matrix of 
$768,432$ rows and five columns which are trend, seasonal, location index, spatial 
smoothed value of original observation, and spatial smoothed value of remainder. 
\item[Map] An intermediate key-value pair is generated for each row of each input
value matrix. The key is the location index for the corresponding row of the 
input matrix, and the value is the one-row matrix with the month index added.
\item[Reduce] Intermediate key-value pairs that share the same location index are 
sorted and sent to one Reduce, then merged together by calling \texttt{rbind}
function. The final by location division is saved on HDFS.
\end{description}

This last MapReduce job is another job in which more attention should be paid.
By tuning potential Hadoop parameters, minimal amount of data is written to the
local disk during the shuffle and sort stage. Also, the efficiency of copy stage 
in which the intermediate outputs are copied to Reduce can also be improved by
appropriated tunning.

In the next section, we are going to discuss the Hadoop parameters considered to
be tuned to improve the performance of the modeling routine. The whole routine is
consist of 7 steps, each of which is done in one MapReduce job. Since the parameters
we are considering are specific targeting the shuffle and sorting stage of the
job, MapReduce job with only Map stage is immune to those Hadoop parameters.
In the modeling routine, the jobs \ref{sec:spaofit}, ~\ref{sec:stlfit}, and 
\ref{sec:sparfit} are this type of MapReduce job which only has Map step. 
Therefore, we do not include them in the experiment of performance since tunning
those Hadoop parameters will not affect their performance at all. In summary, 
the MapReduce jobs we are going to include in each run of the experiment are
\ref{sec:readin}, \ref{sec:swaptoloc}, \ref{sec:swaptotime}, \ref{sec:swaptoloc2}.


\section{Experiment Design}

\subsection{Hadoop User Tuning Parameters}

\subsubsection{\texttt{mapreduce.task.io.sort.mb}}

This parameter controls the size of memory buffer, in megabytes, which is used 
to hold the map output in each Mapper. The value for it is integer varies from 1 
to 2047. The default value for this parameter is 100, which only allocates 100MB
of memory from heap size of JVM for saving map output. It is quite small in 
general. For jobs like reading in and swapping, it is definitely worth to increase
the value of this parameter to give more memory for holding output of each Mapper.

\subsubsection{\texttt{mapreduce.map.sort.spill.percent}}

This parameter works with \texttt{mapreduce.task.io.sort.mb} collectively to 
control the memory used for holding map output. Concretely, a circular memory 
buffer is allocated for each mapper to write intermediate output to. When the 
map output occupied \texttt{mapreduce.map.sort.spill.percent} percent of \\
\texttt{mapreduce.task.io.sort.mb}, the output is spilled to local disk of node
where Mapper is running. Meanwhile the Mapper is keeping writing output to this
circular memory buffer when spilling is proceeding. However if the memory buffer
is filled up during this time period, the Mapper is paused until the spill is
finished. Clearly, it is critical to set this parameter as well as the total 
amount of memory buffer to be high enough if there are numerous amount of output
generated by each Mapper. However, we do not want set them to be too large.
Because the memory buffer is still belongs to the Mapper JVM heap size, which is
decided by \texttt{mapreduce.map.java.opts}. Reserving too much memory from the
total heap size of JVM will leave limited memory usage for other processes sponsed
by the JVM, and indeed will force JVM to involve more garbage clean, which in 
turn will hurt the job performance.    

\subsubsection{\texttt{mapreduce.task.io.sort.factor}}

Every time when the contents in the memory buffer specified by \\
\texttt{mapreduce.task.io.sort.mb} reaches the spill percent threshold, a new
spill file is generated. So there may be multiple output files generated after 
the Map is finished. All of these spill files are merged into one sorted and
partitioned file. This is done in in rounds, and the number of files merged in 
each round is controlled by \texttt{mapreduce.task.io.sort.factor}. The default 
value is 10.

\subsubsection{\texttt{mapreduce.job.reduce.slowstart.completedmaps}}

The parameter controls when Reducers should be launched. Specifically, it specifies
the fraction of the number of Mappers which should be complete before Reducers
are scheduled for the job. Under MapReduce2 (YARN) \cite{YARN} framework, this parameter
becomes critical if the Map of MapReduce job is time consuming. Setting this 
parameter to be a low value as default, which is 0.05, can start the Reducers
doing nothing but waiting for the output from Mappers. However this waist several
containers assigned for Reducers without doing anything instead of assigning them
to the Mappers under pending statues.   

\subsubsection{\texttt{mapreduce.reduce.shuffle.parallelcopies}}

After the map step is finished, the intermediate key-value pairs are partitioned
and sitting on the local disk of the node where Mapper ran. Then each Reducer 
the corresponding partition from all Mapper outputs through the network. And
actually each Reducer evokes a number of copy threads, which controlled by this
parameter, to fetch the intermediate output of Mapper in parallel. This parameter
should be set wisely. Too large value of copy threads will waste for the CPU,
but too small number of copy threads will slow down the copy step of Reducer.
The default value is 5.

\subsubsection{\texttt{mapreduce.reduce.shuffle.input.buffer.percent}}

During the copying stage, the output of Map is copied to part of the total heap 
space of Reducer's JVM. This parameter controls how large the proportion is.
The default value of this parameter is 0.7. Suppose the total heap size of the
JVM of Reducer is 4GB, then 2.8 GB of JVM's memory is used for holding the output
copied from the local disk of nodes ran Mappers.

\subsubsection{\texttt{mapreduce.reduce.shuffle.merge.percent}}

Similar with the Mapper, there is also a spilling mechanism to avoid memory 
overflow. Specifically, the Map output is consistently copied to the proportion 
of heap space specified by \texttt{mapreduce.reduce.shuffle.input.buffer.percent}.
Once the contents in the memory buffer reaches threshold controlled by two 
parameters collectively, the intermediate outputs or the inputs to Reduce is 
spilled to local disk of the node which runs the Reduce. One of the two parameters 
is \\
\texttt{mapreduce.reduce.shuffle.merge.percent} which controls the threshold in 
term of percent of buffer size. If the size of buffer size is occupied over this
proportion, spilling to disk will be triggered. By default, this is set to be 
0.66. Suppose the heap size of Reduce JVM is 4 GB as well, the memory buffer for
coping is 2.8 GB by setting \texttt{mapreduce.reduce.shuffle.input.buffer.percent}
to be 0.7. Then the size threshold is 1.85 GB. Another parameter controls the
threshold is the \texttt{mapreduce.reduce.merge.inmem.threshold} which is discussed
in the following paragraph.

\subsubsection{\texttt{mapreduce.reduce.merge.inmem.threshold}}

Parameter \texttt{mapreduce.reduce.shuffle.merge.percent} controls the memory
buffer threshold in term of the size of contents. Meanwhile the parameter \\
\texttt{mapreduce.reduce.merge.inmem.threshold} controls the contents in memory
buffer in term of counts. So the intermediate output held in the memory buffer
cannot be too many or too large, otherwise they will be spilled to local disk.
The default is 1,000. By setting this parameter to be 0 will hand over the control
of spilling behaviors fully to the \texttt{mapreduce.reduce.shuffle.merge.percent}. 
Notice that the spilled files on local disk are sorted by key. Once all 
intermediate outputs are copied, the Reducers starts to merge and sort all those
spilled files into several sorted files which will be feed to the reduce function
as we defined. This merge process is done in rounds, and the number of merge files
at each round is controlled by \texttt{mapreduce.task.io.sort.factor}, similar 
as in Map stage. Until the number of merged files are equal or less than the number 
specified by \\ \texttt{mapreduce.task.io.sort.factor}, instead of merging them 
into one file, they are directly fed into reduce function. Moreover, not all
of the intermediate data in the memory buffer are spilled into disk. Portion of 
the intermediate data are reserved in the memory, and size is controlled by the
next parameter.

\subsubsection{\texttt{mapreduce.reduce.input.buffer.percent}}

This parameter is specifying the proportion of total heap size of Reducer's JVM
used to reserve the intermediate data in memory and directly feeds the reduce
function. By default, this parameter is set to be 0, which forces all intermediate
data to the local disk and leave all memory for the computation of reduce function.
However, if the memory utilization of reduce computation is extremely light, then
we can increase this parameter to be value close to 1 to keep more intermediate
data in the memory on the reduce side and save strips to the local disk, which
will improve the job performance. In \cite{li2014mronline}, it also suggests to
set this value to be same as \texttt{mapreduce.reduce.shuffle.merge.percent}. 

\section{Software Implementation of R Package}

\subsection{\texttt{Spaloess} Package}

\texttt{Spaloess} is a R package for spatial loess smoothing. It is highly 
depends on the original \texttt{loess} function in the \texttt{stats} package
in base R. There are two main functions in the \texttt{Spaloess} package, which 
are \texttt{spaloess} function for spatial loess smoothing, and \texttt{predloess} 
for the spatial prediction using loess smoothing. Most of implementation in these
two functions are kept the same as the \texttt{loess} and \texttt{predict.loess}
functions, which is R wrapper functions. All memory allocation are done in 
C, and real computation engine is implemented in FORTRAN. \texttt{Spaloess} does
offer the following advantages compared with original \texttt{loess}:

\begin{itemize}
\item Two different distance calculation are available. Euclidean distance and 
Great-Circle distance are allowed. For Great-Circle distance, the input spatial 
attributes must be longitude and latitude degree. Great-Circle distance 
calculation is implemented in FORTRAN.
\item Interpolation kd-tree is built based on all locations instead of only 
non-NA locations as in \texttt{loess} function. 
\item Missing values in the dataset can be handled directly within 
\texttt{spaloess}. 
\end{itemize}

In the original implementation of \texttt{loess} function, the kd-tree 
\cite{bentley1980multidimensional} for 
interpolation is only built based on observations that are not missing values. 
Those missing observations are directly excluded from the analysis. It makes 
extremely harder and computational expensive to predict at those missing value 
especially if those missing value are outside the boundary of the space spanned
by all independent variables. Instead, in the \texttt{spaloess} function, kd-tree 
is built based on all observations even those with missing value. Then 
interpolation can be easily conducted at every location including missing value.

\subsection{\texttt{drSpaceTime} Package}

\texttt{drSpaceTime} is a R package for spatial temporal analysis using divide 
and recombined concept. It is highly depends on three exist R packages: 
\texttt{Rhipe}, \texttt{stlplus}, and \texttt{Spaloess}, which are all open 
source and available on Github \cite{github}. Detailed documentation and examples
can be found in the appendix.

As demonstrated in \ref{sec:routine}, there are seven steps in the modeling 
routine of spatial-temporal data. Each of steps are implemented in a function
from \texttt{drSpaceTime} package, such as \texttt{readIn()} to read in the raw
text files and generating by time division on HDFS; \texttt{spaofit()} to produce
spatial smoothing fit of original observation at each time point; 
\texttt{swaptoLoc()} to generate by location division from the by time division 
including spatial smoothed values; \texttt{stlfit()} to carry out temporal fitting
of the spatial smoothed value in each location by calling \texttt{stlplus} 
function; \texttt{swaptoTime()} to generate by time division from the by location
division which includes all STL fitted components; \texttt{sparfit()} to 
carry out spatial smoothing fit of the remainder component of STL fit.

Besides those functions which represents each steps of the modeling routine, 
there are other two functions \texttt{spacetime.control()} and 
\texttt{mapreduce.control()}, which are both returning a R list object. Concretely,
the \texttt{mapreduce.control()} returns a list including all user tunable 
Hadoop parameters used in a given MapReduce job. \texttt{spacetime.control}, on
the other hand, returns a list with all smoothing parameters needed either for
spatial smoothing or temporal smoothing. 

Finally, the last function in the package is \texttt{predNew()}.