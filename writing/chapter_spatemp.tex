\chapter{Data Analysis of spatial and seasonal-temporal time series with 
Divide and Recombined}

In this chapter, we utilize spatial loess and STL methods under the divide and 
recombined framework to analyze a spatial-temporal data, which is about the monthly
average of daily temperature for the coterminous United States from January 1895
to December 1997.

\input ./spatemp_datasource

\input ./spatemp_exploanaly

\input ./spatemp_a1950

%\section{Monthly Maximum Temperature without Missing Value}%

%Before we jump into the analysis of spatial and temporal dimension interactively, 
%we start our analysis with the time series analysis on each station independently.
%In this section we consider the stations without missing value, which
%there are 432 stations in total. The for visualization purpose, we randomly sample
%100 stations out of them. Their locations are shown as in Figure 
%\href{../plots/100stations.pdf}{\ref*{100stations}}. For each station, we use 
%STL (Seasonal Trend Loess) method to analysis the time series. 
%Additionally, we would like to carry out the analysis for each station in parallel. %

%\begin{framed}
%\begin{center}
%  \href{../plots/100stations.pdf}{Link to figure}
%  \captionof{figure}{The location of the 100 stations}
%  \label{100stations}
%\end{center}
%\end{framed} %

%\subsection{Division by Station}
%\label{sec:divibyStation}%

%In order to analyze multiple time series in parallel, we need a database or division,
%which groups observations of one station all together. And then STL method is 
%applied to each station independently. Finally, the results of decomposition from
%STL method needs to be readily presented by visualization method so that we can
%compare the results cross multiple stations. %

%Fortunately, by using RHIPE, which integrates the R and Hadoop, the division by 
%station and the STL analysis on each station can be handily accomplished.
%On the one hand, R provides analysis friendly data structure and powerful 
%statistical analysis and visualization packages such as STL+, lattice package.
%On the other hand, Hadoop enables the efficient processing of divisions, such as
%subsets of division by station can be represented as key-value pairs saved on 
%HDFS. In the following, we describe in details the procedures to construct the 
%division by station database via RHIPE on a cluster of 11 servers and 242 processors.%

%\begin{description}
%  \item[Input] 10,042,500 key-value pairs from the initial database, with 8,125 
%  \texttt{station.id} and 1,236 months as the key, and a one-row R data.frame 
%  as the value. The value includes (1) \texttt{station.name}, the name of station, 
%  (2) \texttt{longitude}, degree of longitude of the station, (3) \texttt{latitude}, 
%  the latitude of the station, (4) \texttt{elevation}, the meter of elevation the 
%  station, (5) \texttt{year}, year of the observation, (6) \texttt{month}, month 
%  of year (January to December), (7) \texttt{tmax}, the monthly maximum temperature.
%  \item[Output] division by station in form of 432 key-value pairs, with 
%  \texttt{station.id} as the key, and R data.frame containing 1,236 observations
%  information of corresponding station as the value.
%  \item[Map]For every input key-value pair, we do not carry out any adjustment on
%  the one-row data.frame of value. But we do change the key from a vector of 
%  \texttt{station.id} and month index to \texttt{station.id} only. 
%  \item[Reduce] 1,236 of one-row data.frame corresponding to one station are 
%  aggregated to be one data.frame. Concretely all one-row data.frame who share
%  the same station ID are shuffled and transferred to one reducer, and then R function 
%  \texttt{rbind} is called in that reducer to combine those one-row data.frame, 
%  which includes (1) \texttt{year}, year of the observation, (2) \texttt{month}, 
%  month of year (January to December), (3) \texttt{tmax}, the monthly maximum 
%  temperature. The rest of information about the station, \texttt{station.name}, 
%  \texttt{latitude}, \texttt{longitude}, \texttt{elevation}, is save as an 
%  attributes associated with the data.frame named \texttt{location}.
%\end{description}%

%\subsection{Time Series Analysis of Each Subset}%

%After we created the database of division by station, we are ready for the parallel 
%analysis of all subsets of the database. The analysis method we apply to each station
%is Seasonal Trend Loess (SLT). Specifically, we utilize R package named stl2\cite{stl2},
%which provides several enhancements including the ability to deal with missing values 
%and higher order polynomial smoothing compared to the stl method that ships with 
%base R. Then we collect the analysis result of 100 randomly sampled stations together 
%to carry out the visualization. We detail the process of corresponding MapReduce 
%job below.%

%\begin{description}
%  \item[Input] 432 key-value pairs from the division by station database, with 432 
%  unique \texttt{station.id} as the key, and a R data.frame as the value. The value 
%  includes (1) \texttt{year}, year of the observation, (2) \texttt{month}, month 
%  of year (January to December), (3) \texttt{tmax}, the monthly maximum temperature.
%  And an attributes named \texttt{location} encompassing \texttt{station.name},
%  \texttt{latitude}, \texttt{longitude}, and \texttt{elevation} is attached to the
%  data.frame.
%  \item[Output] One key-value pair with 1 as the key, and R data.frame with 123,600
%  rows and 11 columns, which contains \texttt{tmax}, seasonal components, trend 
%  components, and remainders, \texttt{year}, \texttt{month}, and station location 
%  information: \texttt{latitude}, \texttt{longitude}, \texttt{elevation}. 
%  \item[Map]For every input key-value pair, we first order the 1,236 observations 
%  of one station by month index from 1 to 1,236. Then we apply STL method on the 
%  time series of maximum temperature. The fitting result is also saved in a 
%  data.frame. Next, we add a new column to the data.frame which is the corresponding 
%  \texttt{station.id}. Next, we filter out key-value pairs whose key 
%  \texttt{station.id} is not belong in that 100 stations. Finally, all keys are 
%  replaced to be 1 so that fitting results of all 100 stations can be accumulated 
%  into one data.frame in the reducer. \item[Reduce] 100 intermediate key-value 
%  pairs who share the same key 1 are transfered into one reducer. In the reduce 
%  function, they are combined together by calling R function \texttt{rbind} to 
%  generate the final one data.frame. 
%\end{description}%

%As shown in the reduce function above, a set of predefined parameter setting for 
%STL and a string vector of the 100 sampled \texttt{station.id} are passed into 
%the MapReduce job by utilizing \texttt{parameters} argument in \texttt{rhwatch} 
%function, which is used to share R objects in the Global Environment of R session 
%on the front-end server to all workers of the MapReduce job. After the fitting, 
%we would like to visually check the model validation, with same parameter setting, 
%across all different stations.%

%\subsubsection{First Run}%

%In the first run of parallel STL fit, we specify the parameters to be: 
%\texttt{t.window} is 495, \texttt{t.degree} is quadratic, \texttt{s.window} is 
%77, and \texttt{s.degree} is linear, \texttt{inner} is 10. The first thing we would 
%like to check about with respect to the STL fitting is comparing the fitted value 
%with the raw observations. In Figure 
%\href{../plots/100stations/first_run/fitted.100stations.tmax.pdf}
%{\ref*{firstrun.fitted.100stations}}, each page is one station. The whole time 
%series with 1,236 months for one station is chunked into 9 periods. Each of the 
%first 8 periods has 144 monthly observations, the last period has 84 monthly 
%observations. The raw observations are drawn with blue points, and the seasonal 
%component plus trend component are drawn with magenta curve.  %

%\begin{framed}
%\begin{center}
%  \href{../plots/100stations/first_run/fitted.100stations.tmax.pdf}{Link to figure}
%  \captionof{figure}{The fitted value vs. month}
%  \label{firstrun.fitted.100stations}
%\end{center}
%\end{framed}%

%Clearly, no matter where the station is, the fitted values (seasonal+trend) are 
%very close to the true values, and also they capture the seasonal pattern in the 
%raw data. Next, we carry out the diagnostics for each components based on 
%visualization.%

%\textbf{Trend Components}%

%In Figure \href{../plots/100stations/first_run/trend.100stations.tmax.pdf}
%{\ref*{firstrun.trend.100stations}}, the trend component is drawn against to month
%index in blue curve conditional on \texttt{station.id}. So each panel is about 
%the trend component of one station. Meanwhile, we plot with trend components 
%the moving average of yearly mean of monthly maximum temperature.
%Collectively, the same STL parameter setting still can capture the different trend
%behavior across all stations.%

%\begin{framed}
%\begin{center}
%  \href{../plots/100stations/first_run/trend.100stations.tmax.pdf}{Link to figure}
%  \captionof{figure}{The trend components vs. month index}
%  \label{firstrun.trend.100stations}
%\end{center}
%\end{framed}%

%\textbf{Seasonal Components}%

%The diagnostics plot for seasonal component is shown in Figure 
%\href{../plots/100stations/first_run/sea+remaind.month.100stations.tmax.pdf}
%{\ref*{firstrun.searemaind.100stations}}, named seasonal-diagnostics plots, which
%helps us decide how much of the variation in the data other than trend component
%should go into the seasonal component and how much into the remainder.%

%\begin{framed}
%\begin{center}
%  \href{../plots/100stations/first_run/sea+remaind.month.100stations.tmax.pdf}
%  {Link to figure}
%  \captionof{figure}{The sum of seasonal components and remainders vs. year}
%  \label{firstrun.searemaind.100stations}
%\end{center}
%\end{framed}%

%Seasonal component and remainders of
%one station are plotted against to year on one page conditional on month of year
%(Jan to Dec). Let $\bar S_k$ be the mean of the seasonal component for the $k$-th 
%month of year. The magenta curve in the panel for the $k$-th month of year is 
%seasonal components subtract out their mean $\bar S_k$. The blue points are seasonal
%components plus remainders for $k$-th month of year minus their mean $\bar S_k$ 
%as well. The reason for subtracting $\bar S_k$ is to center the values on each 
%panel at zero; note that the vertical scales of all panels of all stations are 
%the same so that we can graphically compare the variation of values on different 
%panels.%

%\textbf{Remainders}%

%The first figure about remainder is shown in Figure 
%\href{../plots/100stations/first_run/remainder.month.100stations.tmax.pdf}
%{\ref*{firstrun.remaind.100stations}}. Remainder of each station is graphed against
%to year conditional on month of year. Also a loess smoothing curve with span equal
%to 0.75 is plotted for each month of year. A flat loess smoothing line represents
%the seasonal component does capture reasonable amount of variation in each 
%sub-series.%

%\begin{framed}
%\begin{center}
%  \href{../plots/100stations/first_run/remainder.month.100stations.tmax.pdf}
%  {Link to figure}
%  \captionof{figure}{The remainders vs. year}
%  \label{firstrun.remaind.100stations}
%\end{center}
%\end{framed}%

%\begin{framed}
%\begin{center}
%  \href{../plots/100stations/first_run/QQ.remainder.100stations.tmax.pdf}
%  {Link to figure}
%  \captionof{figure}{The normal quantiles of remainders}
%  \label{firstrun.QQremaind.100stations}
%\end{center}
%\end{framed}%

%\begin{framed}
%\begin{center}
%  \href{../plots/100stations/first_run/remainder.acf.100stations.tmax.pdf}
%  {Link to figure}
%  \captionof{figure}{Autocorrelation functions for remainder term}
%  \label{firstrun.remaindacf.100stations}
%\end{center}
%\end{framed}%

%As in Figure 
%\href{../plots/100stations/first_run/remainder.acf.100stations.tmax.pdf}
%{\ref*{firstrun.remaindacf.100stations}},
%although there are a few slightly significant autocorrelation estimates for some
%of stations, surprisingly the overall behavior is indicative of independence.%
%

%\subsubsection{Second Run}%

%In the first run of parallel STL fit, we specify that parameters to be: 
%\texttt{t.window} is 495, \texttt{t.degree} is quadratic, \texttt{s.window} is 
%77, and \texttt{s.degree} is linear, \texttt{inner} is 10. However we randomly 
%sample another 100 stations to be in our visualization checking because we would 
%like to see if parallel STL fitting is robust cross different parameter setting 
%and different locations.







