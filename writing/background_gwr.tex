\section{Geographically Weighted Regression}

Another critical application of locally weighted regression method is in the 
spatial analysis area which called geographically weighted regression. It was first
introduced by Fotheringham in \cite{fotheringhamgeographically}, and it is nothing
but a local weighted regression method within two dimension spatial space. It is
a method for spatial non-stationary. Compared with other popular spatial analysis
methods like kriging \cite{krige1951statistical} which illustrates the relationship 
between neighbors by distance and a pre-defined covariance matrix, local regression 
procedure is trying to tackle the problem from a different aspect.

\subsection{Basic Methodology of GWR}

Let $y=(y_{(u_1,v_1)}, y_{(u_2,v_2)}, \cdots, y_{(u_n,v_n)})$ be the observation 
at $n$ different locations, each of which has a location coordinate $(u_i, v_i)$ 
in the spatial dimension, like longitude and latitude coordinates. Let design matrix 
and weight matrix for $(u, v)$ to be $X$ and $W(u, v)$ respectively, the estimate 
of regression coefficient $\hat \beta$ is:
\begin{equation}
\hat \beta(u, v) = (X^TW(u,v)X)^{-1}X^TW(u,v)y
\end{equation}
Similarly with general loess smoothing method, the whole procedure of GWR can
also be understood as a linear operation. So the fitted value at $(u, v)$ can 
be written as a weighted average of $y_i$ as
\begin{equation}
\hat y_{(u,v)} = \sum_{i=1}^n l_i(u,v)y_i = l(u,v)y
\end{equation}
where
\begin{equation}
l(u,v) = (l_1(u,v), \cdots, l_n(u,v)) = e_1^T(X^TW(u,v)X)^{-1}X^TW(u,v)
\end{equation}
with $e_1^T = (1,0,\cdots,0)$. So the estimate at all location can be expressed
as
\begin{equation}
\hat y = Ly
\end{equation}
where 
\begin{equation}
L =  
\begin{pmatrix}
  l_1(u_1,v_1) & l_2(u_1,v_1) & \cdots & l_n(u_1,v_1) \\
  l_1(u_2,v_2) & l_2(u_2,v_2) & \cdots & l_n(u_2,v_2) \\
  \vdots  & \vdots & \ddots & \vdots  \\
  l_1(u_n,v_n) & l_2(u_n,v_n) & \cdots & l_n(u_n,v_n) 
\end{pmatrix}
\end{equation}
$L$ is called operator matrix \cite{hafen2010local}, and it is very similar to the
hat matrix in global least-squares regression which we can use to carry out 
statistical inference. But $L$ is neither symmetric nor idempotent.

\subsubsection{Weight Function}

In general, any function satisfying following conditions can be weight function
for the GWR:
\begin{enumerate}
\item $W(\mu) > 0$ for $|\mu| < 1$
\item $W(-\mu) = W(\mu)$
\item $W(\mu)$ is a nonincreasing function for $\mu \ge 0$
\item $W(\mu) = 0$ for $|\mu| \ge 1$
\end{enumerate}
There are several possible functions proposed in \cite{fotheringhamgeographically}.
In fact, most weight functions with smooth contact will give reasonable results.
Consequently, the analysis conducted in chapter two only considers bi-square
weight function, which just follows the default choice of loess procedure.

\subsubsection{Spatial Neighborhood}

In practice, the results of GWR are relatively insensitive to the choice of 
weighting function but they are sensitive to the bandwidth of the particular 
weighting function. There are two different ways to identify neighborhood of 
target fitting point. The first one is called fixed spatial kernel 
\cite{fotheringhamgeographically}. A fixed radius of neighborhood is specified
for every fitting location. 
However, a potential problem coming with this is that for some fitting locations,
where data is sparse, the local models might be calibrated on very few data points, 
giving rise to parameter estimates with high variance issue. 

Accordingly, to reduce the high variance issue, another approach is proposed to 
fix the number of observations in the neighborhood, so called adaptive spatial
kernels \cite{fotheringhamgeographically}, which is the default in the original 
loess method. So for each fitting location, a fixed number $k$ of neighbors is 
specified, and the bandwidth of the weight function is determined by the distance
from the target location to the $k$th nearest neighbor. 
